\documentclass[12pt, letterpaper, titlepage]{article}
\usepackage[left=3.5cm, right=2.5cm, top=2.5cm, bottom=2.5cm]{geometry}
\usepackage[MeX]{polski}
\usepackage[utf8]{inputenc}
\usepackage{graphicx}
\usepackage{enumerate}
\usepackage{amsmath} %pakiet matematyczny
\usepackage{amssymb} %pakiet dodatkowych symboli
\title{Tytuł}
\author{Jakub Puczyłowski}
\date{15 Października 2022 roku.}
\begin{document}
\maketitle
\section{Lista planet i ich księżyców}
\begin{enumerate}[A)]
\item \textbf {Planety}
\begin{enumerate}[1.]
\item \underline{Merkury} 
\item \underline{Wenus} 
\item \underline{Ziemia} 
\begin{enumerate}[a)]
\item \textit{Księżyc}
\end{enumerate}
\item \underline{Mars} 
\item \underline{Jowisz} 
\begin{enumerate}[a)]
\item \textit{Io}
\item \textit{Europa}
\item \textit{Ganimedes}
\item \textit{Kallisto}
\end{enumerate}
\item \underline{Saturn} 
\item \underline{Uran} 
\item \underline{Neptun} 
\end{enumerate}
\end{enumerate}
\newpage
\subsection{Merkury}
\ \ \ \ \ najmniejsza i najbliższa Słońca planeta Układu Słonecznego. Jako planeta dolna znajduje się dla ziemskiego obserwatora zawsze blisko Słońca, dlatego jest trudna do obserwacji. Mimo to należy do planet widocznych gołym okiem i była znana już w starożytności. Merkurego dojrzeć można jedynie tuż przed wschodem lub tuż po zachodzie Słońca.

Ukształtowanie powierzchni Merkurego przypomina Księżyc: są na nim liczne kratery uderzeniowe i pozbawiony jest on atmosfery. W przeciwieństwie do Księżyca, planeta ma jednak duże żelazne jądro, generujące pole magnetyczne stukrotnie słabsze od ziemskiego. Rozmiar jądra sprawia, że Merkury ma jedną z największych gęstości spośród planet Układu Słonecznego (Ziemia ma nieznacznie większą gęstość). Merkury nie ma naturalnych satelitów.

Pierwsze udokumentowane obserwacje Merkurego sięgają pierwszego tysiąclecia p.n.e. Starożytni Grecy początkowo uważali, że są to dwa ciała niebieskie: pierwsze widzialne tylko przed wschodem Słońca (nazywali je Apollo), drugie widzialne tylko po zachodzie Słońca (nazywali je Hermesem). Starożytni Egipcjanie, Chaldejczycy oraz późniejsi astronomowie greccy wiedzieli, że Merkury widoczny o poranku i wieczorem jest tą samą planetą. Było to znane Egipcjanom już około 1150 roku p.n.e. Za sprawą szybkiego ruchu planety, powodowanego jej krótką orbitą, Rzymianie nadali planecie nazwę na cześć posłańca bogów i patrona handlarzy – Merkurego. Symbol astronomiczny planety to stylizowana wersja kaduceusza Hermesa.

W porównaniu z innymi planetami Układu Słonecznego o Merkurym wiadomo stosunkowo niewiele; ze względu na problemy natury technicznej zbadały go dotychczas tylko dwie sondy. Pierwsza z nich – Mariner 10 – w latach 1974–1975 trzykrotnie przeleciała obok Merkurego i wykonała mapy 45\% powierzchni. Sonda MESSENGER w 2008 i 2009 roku dokonała trzech przelotów obok planety, po czym w latach 2011–2015 badała ją z orbity jako sztuczny satelita. Wystrzelona w 2018 roku sonda BepiColombo ma dotrzeć na orbitę wokół Merkurego w 2025 roku.
\newpage
\subsection{Wenus}
\ \ \ \ \ Druga pod względem odległości od Słońca planeta Układu Słonecznego. Jest trzecim pod względem jasności ciałem niebieskim widocznym na niebie, po Słońcu i Księżycu. Ponieważ Wenus jest bliżej Słońca niż Ziemia, zawsze jest widoczna w niewielkiej odległości kątowej od niego.Odległość Wenus od Ziemi zmienia się w zakresie od około 40 mln km do około 259 mln km.Nazwa planety wzięła się od rzymskiej bogini miłości, Wenus. Na niebie planeta jest widoczna przez około trzy godziny przed wschodem Słońca nad wschodnim horyzontem lub po zachodzie Słońca nad zachodnim horyzontem. Nieodłączna towarzyszka wschodzącego i zachodzącego Słońca, nazywana jest także Gwiazdą Poranną (Zaranną, Porankową) lub Jutrzenką (łac. Stella Matutina), kiedy zwiastuje wschód Słońca, albo Gwiazdą Wieczorną, która finalizuje jego zachód.
\newline
\newline 
\ \ \ \ \ Wenus jest klasyfikowana jako planeta skalista (inaczej: typu ziemskiego) i jest czasami nazywana „planetą bliźniaczą” albo „siostrą Ziemi” – ze względu na podobną wielkość, masę i skład chemiczny. Atmosfera Wenus jest jednak zupełnie odmienna od ziemskiej. Jest pokryta nieprzezroczystą warstwą dobrze odbijających światło chmur kwasu siarkowego, które nie pozwalają na obserwację jej powierzchni z kosmosu w świetle widzialnym. Ma najgęstszą atmosferę ze wszystkich planet skalistych w Układzie Słonecznym, składającą się głównie z dwutlenku węgla. Na Wenus nie ma obiegu węgla, który powodowałby wiązanie węgla w skałach. Nie stwierdzono na niej również śladów organizmów żywych, które by go wiązały w biomasie. Istnieją przypuszczenia, że w przeszłości na Wenus były oceany, tak jak na Ziemi, ale odparowały, gdy temperatura powierzchni wzrosła. Obecny krajobraz Wenus jest suchy i pustynny, tworzony przez pokryte pyłem skały. Woda w atmosferze najprawdopodobniej dysocjowała, a z powodu braku pola magnetycznego, wodór został wywiany w przestrzeń międzyplanetarną przez wiatr słoneczny. Ciśnienie atmosferyczne na powierzchni planety jest około 92 razy większe niż na Ziemi.
\newline
\newline 
\ \ \ \ \ Ukształtowanie powierzchni Wenus było przedmiotem spekulacji aż do drugiej połowy XX wieku, gdy zostało zbadane przez sondy Wenera i Magellan. Powierzchnia Wenus została ukształtowana przez zjawiska wulkaniczne, zachodzące w skali znacznie większej niż na Ziemi, a duże stężenie związków siarki w atmosferze wskazuje na trwającą ciągle aktywność wulkaniczną. Jednak brak obserwowanych przepływów lawy w okolicach odkrytych kalder pozostaje zagadką. Jest niewiele widocznych kraterów uderzeniowych, co wskazuje, że powierzchnia jest stosunkowo młoda – ma około 300–600 milionów lat. Nie ma tektoniki płyt, prawdopodobnie dlatego, że jej skorupa jest zbyt sztywna.
\newpage
\subsection{Ziemia}
\ \ \ \ \ Trzecia, licząc od Słońca, oraz piąta pod względem wielkości planeta Układu Słonecznego. Pod względem średnicy, masy i gęstości jest to największa planeta skalista Układu Słonecznego. Ziemia jest zamieszkana przez miliony gatunków, w tym przez człowieka. Jest jedynym znanym miejscem we Wszechświecie, w którym występuje życie. Według danych zebranych metodą datowania izotopowego, planeta uformowała się ok. 4,54 ± 0,05 mld lat temu.
 Prawdopodobnie w ciągu pierwszego miliarda lat po uformowaniu się Ziemi w oceanach pojawiło się życie. Z żyjących na Ziemi organizmów żywych składa się biosfera, która wpływa na jej atmosferę, hydrosferę, litosferę i inne czynniki abiotyczne planety, umożliwiając rozwój i wzrost liczby organizmów aerobowych i anaerobowych oraz powstanie ozonosfery. Rozwój życia na lądzie i w wodzie umożliwiła powłoka ozonowa oraz ziemskie pole magnetyczne, zmniejszając natężenie promieniowania ultrafioletowego, oraz magnetosfera, odbijająca cząstki wiatru słonecznego i promieniowania kosmicznego. \newline
 \newline Dystans dzielący Słońce od Ziemi, jej właściwości fizyczne oraz jej historia geologiczna są najważniejszymi czynnikami, które pozwoliły organizmom żyć i ewoluować. Różnorodność biologiczna Ziemi nieustannie powiększa się, chociaż w dziejach życia Ziemi proces ten był kilkukrotnie przerywany, kiedy miało miejsce masowe wymieranie gatunków. Pomimo że naukowcy szacują, że ok. 99\% gatunków organizmów żywych (ok. 5 mld) kiedykolwiek zamieszkujących Ziemię uważa się za wymarłe, wciąż mieszka na niej ok. 10–14 mln gatunków, z czego 1,2 mln zostało udokumentowanych. \newline
 \newline Litosfera Ziemi dzieli się na kilkanaście płyt tektonicznych, które przesuwają się względem siebie w ciągu okresów trwających nawet przez wiele milionów lat. W ich trakcie dochodzi do znacznej zmiany położenia kontynentów. Powierzchnię w 70,8\% zajmuje woda wszechoceanu zawarta w morzach i oceanach; pozostałe 29,2\% stanowią kontynenty i wyspy, na powierzchni których znajdują się jeziora oraz inne źródła wody tworzące hydrosferę. Niezbędnej do życia na Ziemi wody w stanie ciekłym nie wykryto na powierzchni innych ciał niebieskich. Wnętrze Ziemi pozostaje aktywne; składa się z grubego i w dużej mierze stałego płaszcza, płynnego jądra zewnętrznego (generującego pole magnetyczne) oraz składającego się z żelaza stałego jądra wewnętrznego. Strefy podbiegunowe Ziemi są pokryte lodem wchodzącym w skład pokrywy lodowej Antarktydy (biegun południowy), pokrywy lodowej Grenlandii i lodu morskiego, w tym arktycznego paku lodowego (biegun północny). Ziemia oddziałuje grawitacyjnie z innymi ciałami w przestrzeni kosmicznej, zwłaszcza ze Słońcem i Księżycem. Planeta wykonuje jedno okrążenie wokół Słońca raz na każde 365,256 obrotów wokół własnej osi. Czas jednego okrążenia wokół Słońca nazywa się rokiem gwiazdowym i odpowiada 365,256 dniom czasu słonecznego. Nachylenie osi Ziemi do prostej prostopadłej do płaszczyzny orbity wynosi 23,44°, co prowadzi do rocznych wahań oświetlenia, które powodują m.in. występowanie na jej powierzchni pór roku, które tworzą rok zwrotnikowy. Wokół Ziemi krąży jeden naturalny satelita – Księżyc. 
\subsubsection{Księżyc}
\ \ \ \ \ Jedyny naturalny satelita Ziemi (nie licząc tzw. księżyców Kordylewskiego, które są obiektami pyłowymi i przez niektórych badaczy uważane za obiekty przejściowe). Jest piątym co do wielkości księżycem w Układzie Słonecznym. Przeciętna odległość od środka Ziemi do środka Księżyca to 384 399 km, co stanowi mniej więcej trzydziestokrotność średnicy ziemskiej. Średnica Księżyca wynosi 3474 km, nieco więcej niż 1/4 średnicy Ziemi. Oznacza to, że objętość Księżyca wynosi około 1/50 objętości kuli ziemskiej. Przyspieszenie grawitacyjne na jego powierzchni to 1,622 m/s², a to 1/6 przyspieszenia na Ziemi.
\newline
\newline
Księżyc odbija światło emitowane przez Słońce.

Księżyc wykonuje pełny obieg wokół Ziemi w ciągu 27,3 dnia (tzw. miesiąc syderyczny), a okresowe zmiany w geometrii układu Ziemia–Księżyc–Słońce powodują występowanie powtarzających się w cyklu 29,53-dniowym (tzw. miesiąc synodyczny) faz Księżyca.

Księżyc to jedyne ciało niebieskie, do którego podróżowali i na którym wylądowali ludzie. Do tej pory na księżycowym globie stanęło dwunastu astronautów. Pierwszym sztucznym obiektem w historii, który przeleciał blisko Księżyca, była wystrzelona przez Związek Radziecki sonda kosmiczna Łuna 1; Łuna 2 jako pierwszy statek osiągnęła powierzchnię ziemskiego satelity, zaś Łuna 3 jeszcze w tym samym roku, co poprzedniczki – 1959 – wykonała pierwsze zdjęcia niewidocznej z Ziemi strony Księżyca. Pierwszym statkiem, który przeprowadził udane miękkie lądowanie była Łuna 9, zaś pierwszym bezzałogowym pojazdem umieszczonym na orbicie Księżyca – Łuna 10 (oba w 1966). Amerykański program Apollo obejmował misje załogowe, zakończone 6 lądowaniami w latach 1969–1972. Eksploracja powierzchni Księżyca przez ludzi została przerwana wraz z zakończeniem lotów Apollo, ostatnią misją bezzałogową była radziecka Łuna 24 w 1976 roku. Dopiero w 2013 roku na powierzchni Księżyca wylądowała chińska sonda Chang’e 3 z łazikiem Yutu.
\newpage
\subsection{Mars}
\ \ \ \ \ Czwarta od Słońca planeta Układu Słonecznego. Krąży między orbitą Ziemi a pasem planetoid, dzielącym go od orbity Jowisza. Planeta została nazwana od imienia rzymskiego boga wojny – Marsa, zawdzięcza ją barwie, która przy obserwacji z Ziemi wydaje się rdzawo-czerwona i kojarzyła się starożytnym Rzymianom z pożogą wojenną. Odcień ten bierze się od tlenków żelaza pokrywających powierzchnię. Mars jest planetą wewnętrzną z cienką atmosferą, o powierzchni usianej kraterami uderzeniowymi, podobnie jak powierzchnia Księżyca i wielu innych ciał Układu Słonecznego. Występują na nim różne rodzaje terenu, podobne do ziemskich: wulkany, doliny, kaniony, pustynie i polarne czapy lodowe. Okres obrotu wokół własnej osi jest niewiele dłuższy niż ziemski i wynosi 24,6229 godziny (24 h 37 m 22 s). Na Marsie znajduje się najwyższy wulkan w Układzie Słonecznym – Olympus Mons i największy kanion – Valles Marineris. Gładki obszar równinny Vastitas Borealis na półkuli północnej, który obejmuje 40\% powierzchni planety, może być pozostałością ogromnego uderzenia. W przeciwieństwie do Ziemi, Mars jest mało aktywny geologicznie i nie ma tektoniki płyt.
\newpage
\subsection{Jowisz}
\ \ \ \ \  Piąta w kolejności od Słońca i największa planeta Układu Słonecznego[b]. Masa Jowisza jest nieco mniejsza niż jedna tysięczna masy Słońca, a zarazem dwa i pół razy większa niż łączna masa pozostałych planet w Układzie Słonecznym. Wraz z Saturnem, Uranem i Neptunem tworzą grupę gazowych olbrzymów, nazywaną czasem również planetami jowiszowymi.


Planetę znali astronomowie w czasach starożytnych, była związana z mitologią i wierzeniami religijnymi wielu kultur. Rzymianie nazwali planetę na cześć najważniejszego bóstwa swojej mitologii–Jowisza. Jest to trzeci co do jasności naturalny obiekt na nocnym niebie po Księżycu i Wenus (okresowo, w momencie wielkiej opozycji, jasnością może mu dorównywać Mars).

Największa planeta Układu Słonecznego składa się w trzech czwartych z wodoru i w jednej czwartej z helu; może posiadać także skaliste jądro złożone z cięższych pierwiastków. Szybka rotacja nadaje mu kształt spłaszczonej elipsoidy obrotowej (ma też niewielkie, ale zauważalne zgrubienie w płaszczyźnie równika). Powierzchnię planety, którą stanowią nieprzezroczyste wyższe warstwy atmosfery, pokrywa kilka warstw chmur, układających się w charakterystyczne pasy widoczne z Ziemi. Najbardziej znanym szczegółem jego powierzchni jest odkryta w XVII wieku przy pomocy teleskopu Wielka Czerwona Plama, będąca antycyklonem o średnicy większej niż średnica Ziemi. Planeta ma słabo widoczne pierścienie i potężną magnetosferę. Jest znanych 79 księżyców Jowisza. Cztery największe, zwane galileuszowymi, odkrył Galileusz w 1610. Ganimedes, największy z księżyców, ma średnicę większą niż planeta Merkury.

Planeta była wielokrotnie badana przez sondy, zwłaszcza na początku programu Pioneer i programu Voyager, a następnie przez sondę Galileo. Od lipca 2016 roku na orbicie dookoła planety znajduje się sonda kosmiczna Juno, która miała okrążać ją i badać do lipca 2021, misję sondy przedłużono do września 2025.
\subsubsection{Io}
\ \ \ \ \ Trzeci co do wielkości księżyc Jowisza, z grupy księżyców galileuszowych, czwarty co do wielkości w Układzie Słonecznym. Charakteryzuje się niezwykle silną aktywnością wulkaniczną.

Wszystkie księżyce galileuszowe można bez trudu dostrzec przez zwyczajną lornetkę. W bezchmurne noce osoby z bardzo ostrym wzrokiem są w stanie zobaczyć je nieuzbrojonym okiem (księżyce galileuszowe mają jasność poniżej 6 magnitudo, wartości stanowiącej możliwość graniczną obserwacji ciał niebieskich dla ludzkiego oka).
\subsubsection{Europa}
\ \ \ \ \ Czwarty co do wielkości księżyc Jowisza z grupy księżyców galileuszowych i szósty co do wielkości satelita w Układzie Słonecznym. Pod jego lodową skorupą znajduje się prawdopodobnie ocean ciekłej wody.
\subsubsection{Ganimedes}
\ \ \ \ \ Największy księżyc Jowisza, należący do grupy księżyców galileuszowych. Jest równocześnie największym znanym księżycem w Układzie Słonecznym, ma większą średnicę od Merkurego, najmniejszej planety w Układzie Słonecznym.

Wszystkie księżyce galileuszowe można dostrzec przez zwyczajną lornetkę.
\subsubsection{Kallisto}
\ \ \ \ \  Drugi co do wielkości księżyc Jowisza, trzeci w Układzie Słonecznym, najbardziej oddalony z księżyców galileuszowych.

Kallisto jest o 1\% mniejsza od Merkurego, lecz około trzykrotnie od niego lżejsza.  W przeciwieństwie do trzech pozostałych, nie jest częścią rezonansu orbitalnego i nie jest w tak znacznym stopniu poddany sile pływowej[2]. Kallisto obraca się synchronicznie i zawsze jest zwrócona do Jowisza tą samą stroną. W przeciwieństwie do wewnętrznych satelitów, jest mniej narażona na działanie magnetosfery gazowego giganta[3].

Kallisto jest księżycem lodowym, składa się w przybliżeniu z równej ilości skał i lodu. . Na powierzchni metodą spektroskopową wykryto lód wodny, dwutlenek węgla, krzemiany oraz związki organiczne. Obserwacje sondy Galileo wskazują, że Kallisto może mieć małe krzemianowe jądro i prawdopodobnie ocean ciekłej wody na głębokości ponad 100 km[4][5].

Księżyc był badany przez różne sondy kosmiczne: Pioneer 10 i 11 oraz Galileo i Cassini.
\newpage
\subsection{Saturn}
\ \ \ \ \ Gazowy olbrzym, szósta planeta Układu Słonecznego pod względem odległości od Słońca, druga po Jowiszu pod względem masy i wielkości. Charakterystyczną jego cechą są pierścienie, składające się głównie z lodu i w mniejszej ilości z odłamków skalnych; inne planety-olbrzymy także mają systemy pierścieni, ale żaden z nich nie jest tak rozległy ani tak jasny. Według danych z października 2019 roku znane są 82 naturalne satelity Saturna[3], co czyni go liderem wśród planet z największą liczbą księżyców.

Promień Saturna jest około 9 razy większy od promienia Ziemi[4]. Chociaż jego gęstość to tylko jedna ósma średniej gęstości Ziemi, ze względu na wielokrotnie większą objętość masa Saturna jest dziewięćdziesiąt pięć razy większa niż masa Ziemi[5].

We wnętrzu Saturna panują ciśnienie i temperatura, których nie udało się dotąd uzyskać w laboratoriach na Ziemi. Wnętrze gazowego olbrzyma najprawdopodobniej składa się z jądra z żelaza, niklu, krzemu i tlenu, otoczonego warstwą metalicznego wodoru, warstwy pośredniej ciekłego wodoru i ciekłego helu oraz zewnętrznej warstwy gazowej[6]. Prąd elektryczny w warstwie metalicznej wodoru generuje pole magnetyczne Saturna, które jest nieco słabsze niż pole magnetyczne Ziemi i ma około jedną dwudziestą natężenia pola wokół Jowisza[7]. Zewnętrzna warstwa atmosfery wydaje się na ogół spokojna, choć mogą się na niej utrzymywać długotrwałe układy burzowe. Na Saturnie wieją wiatry o prędkości ok. 1800 km/h; są one silniejsze niż na Jowiszu.

Saturn ma 9 pierścieni, składających się głównie z cząsteczek lodu, a także ze skał i pyłu kosmicznego. Potwierdzono odkrycie 82 księżyców krążących po orbicie planety, spośród których 53 mają oficjalne nazwy[8]. Do tego dochodzą setki „księżyców karłowatych” w pierścieniach planetarnych. Jego księżyc Tytan to drugi co do wielkości księżyc w Układzie Słonecznym (po księżycu Jowisza Ganimedesie), jest większy od planety Merkury i jest jedynym księżycem w Układzie Słonecznym posiadającym gęstą atmosferę[9].
\newpage
\subsection{Uran}
\ \ \ \ \ Gazowy olbrzym, siódma od Słońca planeta Układu Słonecznego, trzecia pod względem wielkości i czwarta pod względem masy. Nazwa planety pochodzi od Uranosa, greckiego boga, ojca Kronosa (Saturna) i dziada Zeusa (Jowisza). Choć jest widoczny gołym okiem[b] podobnie jak pięć innych planet, umknął uwadze starożytnych obserwatorów z powodu małej jasności i powolnego ruchu po sferze niebieskiej[3]. William Herschel ogłosił odkrycie planety 13 marca 1781, po raz pierwszy w historii rozszerzając znane granice Układu Słonecznego. Uran to również pierwsza planeta odkryta przy pomocy teleskopu.

Uran budową i składem chemicznym przypomina Neptuna, a obie planety mają odmienną budowę i skład niż większe gazowe olbrzymy: Jowisz i Saturn. Astronomowie czasem umieszczają je w oddzielnej kategorii „lodowych olbrzymów”. Atmosfera Urana, chociaż składa się głównie z wodoru i helu (podobnie jak atmosfery Jowisza i Saturna), zawiera więcej zamrożonych substancji lotnych (tzw. lodów) niż atmosfery większych planet-olbrzymów; są to substancje takie jak woda, amoniak i metan oraz śladowe ilości węglowodorów[4]. Ma ona złożoną, warstwową strukturę. Uważa się, że jej najniższe chmury tworzy woda, a najwyższa warstwa chmur jest utworzona z kryształków metanu[4]. Z kolei wnętrze Urana składa się głównie z lodów i skał[5].

Podobnie jak inne planety-olbrzymy, Uran posiada system pierścieni, magnetosferę i liczne księżyce. System Urana ma unikatową konfigurację wśród planet, ponieważ jego oś obrotu jest silnie nachylona i znajduje się prawie w płaszczyźnie orbity planety. W tej sytuacji jego biegun północny i południowy leżą tam, gdzie równik większości innych planet[6]. Widziane z Ziemi pierścienie Urana czasami układają się wokół planety jak tarcza łucznicza, zaś księżyce planety krążą wokół niej jak wskazówki zegara, choć w 2007 i 2008 pierścienie planety były ustawione krawędzią do osi obserwacji. W 1986 obrazy z sondy Voyager 2 pokazały Urana jako planetę praktycznie pozbawioną wyróżniających się cech powierzchni w świetle widzialnym, bez pasm chmur i burz podobnych do istniejących na pozostałych planetach-olbrzymach[6]. Jednak w ostatnich latach obserwacje prowadzone z Ziemi ukazały oznaki zmian pór roku i zwiększonej aktywności zjawisk pogodowych, gdy Uran zbliżył się do równonocy. Prędkość wiatru na Uranie może osiągnąć 250 metrów na sekundę.
\newpage
\subsection{Neptun}
\ \ \ \ \ Gazowy olbrzym, ósma, najdalsza od Słońca planeta w Układzie Słonecznym, czwarta pod względem średnicy i trzecia pod względem masy. Neptun jest ponad 17 razy masywniejszy od Ziemi i trochę masywniejszy od swojego bliźniaka, Urana, który ma masę prawie 15 razy większą od masy Ziemi[b]. Krąży wokół Słońca w odległości około 30 razy większej niż dystans Ziemia-Słońce. Nazwa pochodzi od rzymskiego boga mórz Neptuna. Jego symbol astronomiczny to Neptune symbol.svg, stylizowany trójząb Neptuna.

Odkryty 23 września 1846[4] Neptun jest jedyną planetą Układu Słonecznego, której istnienie wykazano nie na podstawie obserwacji nieba, ale na drodze obliczeń matematycznych. Niespodziewane zmiany w ruchu orbitalnym Urana doprowadziły astronomów do wniosku, że podlega on perturbacjom pochodzącym od nieznanej planety. Neptun został następnie zaobserwowany przez Johanna Galla w miejscu przewidzianym przez Urbaina Le Verriera, a wkrótce został też odkryty jego największy księżyc, Tryton; żaden z pozostałych 13 znanych dziś księżyców Neptuna nie został odkryty za pomocą teleskopu aż do XX wieku. Neptun został odwiedzony przez tylko jedną sondę kosmiczną, Voyager 2, która przeleciała w pobliżu planety 25 sierpnia 1989.

Neptun przypomina składem Urana, co odróżnia je od większych gazowych olbrzymów, Jowisza i Saturna. Atmosfera Neptuna, choć – podobnie jak na Jowiszu i Saturnie – składa się głównie z wodoru i helu wraz ze śladami węglowodorów i prawdopodobnie azotu, zawiera większą ilość tzw. „lodów”, czyli substancji lotnych w warunkach ziemskich, takich jak woda, amoniak i metan. Astronomowie czasami kategoryzują Urana i Neptuna jako „lodowe olbrzymy” w celu podkreślenia tych różnic[5]. Wnętrze Neptuna, podobnie jak Urana, składa się głównie z lodów i skał[6]. Ślady metanu w zewnętrznych obszarach planety przyczyniają się do nadania jej charakterystycznego niebieskiego koloru[7].

W przeciwieństwie do niemal pozbawionej wyróżniających się struktur atmosfery Urana, atmosferę Neptuna cechuje aktywność i widoczne układy pogodowe. Podczas przelotu w 1989 roku Voyager 2 odkrył na półkuli południowej Wielką Ciemną Plamę, porównywalną z Wielką Czerwoną Plamą na Jowiszu. Takie struktury są napędzane przez najsilniejsze wiatry w Układzie Słonecznym; rekord prędkości wiatru to aż 2100 km/h[8]. Duża odległość od Słońca powoduje, że zewnętrzna atmosfera Neptuna jest jednym z najzimniejszych miejsc w Układzie Słonecznym.

Neptun ma słaby i pozornie niekompletny system pierścieni. Pierwsze sygnały o istnieniu tych struktur pochodzą z lat 60. XX w., ale dopiero w 1989 roku sonda Voyager 2 bezsprzecznie potwierdziła ich istnienie[11].

\end{document}